\documentclass[a4paper,10pt]{report}
\usepackage[utf8]{inputenc}
\usepackage[pdftex,colorlinks]{hyperref}
    \hypersetup{%
    pdftitle={Kunjika : key for open source question answer application},
    pdfauthor={Shashi Kant},
    pdfsubject={kunjika, open source question answer application}
    pdfkeywords={kunjika, shashi kant, open source, application, question,answer,jinja,flask,couchbase,},
    bookmarksnumbered,
    pdfstartview={FitH},
    urlcolor=webblue,
    linkcolor=webgreen,
}%

% Title Page
\title{Kunjika : key for open source question answer application}
\author{shashi kant}
\date{\today \\ \texttt{shashikant@tenhash.com}}


\begin{document}
\maketitle
\begin{abstract}
Asking questions and providing solutions are integral and essential parts of learning process and to help in this, there exist applications like stackoverflow, AnswerHub, Discourse, Question2Answer, OSQA, Shapado etc. Such applications can extend help in educational institutions, company or organization. Sometimes such web apps are public and hosted on public domain while sometimes it is required to keep it private to the members of organization or institution depending on the nature of discussion.

Reason for developing Kunjika is lack of good quality free \& open source QA applications. Most applications take monthly charges depending on the uses and don't provide source code. Good open source application like OSQA is now no longer being developed and rest of open source applications lacks features which were needed by us and many times they were in PHP which is an unknown territory for us.

We have kept the future demands in mind while designing the structure and selecting the technologies to be used. We are  using Python and Fask.  We are using Couchbase which is a non-relational and JSON document based database. It is high performance, easily scalable and has build-in object level cache which makes it suitable for Kunjika. It also provides Memcached functionality.
\end{abstract}

\end{document}
